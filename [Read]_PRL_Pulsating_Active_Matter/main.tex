\documentclass[10pt,aspectratio=43,mathserif,table]{beamer} 
%  设置为 Beamer 文档类型,设置字体为 10pt,长宽比为16:9,数学字体为 serif 风格
\batchmode
\usepackage{graphicx}
\usepackage{subfigure}
\usepackage{fontspec}

\setmainfont{Harding Text Web Regular Regular.ttf}
\usepackage{diagbox} % 表头斜线分区
\usepackage{unicode-math}
\usefonttheme{serif}
\setmathrm{Harding Text Web Regular Regular.ttf} % 设置数学字体为 Times New Roman
\setmathfont{TeX Gyre Termes Math} % 如果您使用 XeLaTeX 或 LuaLaTeX 编译,可以使用其他数学字体
\setmathtt{Courier New} % 设置等宽字体为 Courier New
\setboldmathrm{Times New Roman}
\setmathfont{TeX Gyre Termes Math}[version=bold] % 设置粗体数学字体
\setmathfont{TeX Gyre Termes Math}[range={\mathit}]


\usetheme{Berlin} %主题
\setbeamertemplate{page number in head/foot}[pagenumber]
%\usecolortheme{sustech} %主题颜色

\usepackage[ruled,linesnumbered]{algorithm2e}

\usepackage{fancybox}
\usepackage{xcolor}
\usepackage{listings}
\usepackage{multicol}
\usepackage{booktabs}
\usepackage{colortbl}

\newcommand{\Console}{Console}
\lstset{ %
	backgroundcolor=\color{white},   % choose the background color
	basicstyle=\footnotesize\rmfamily,     % size of fonts used for the code
	columns=fullflexible,
	breaklines=true,                 % automatic line breaking only at whitespace
	captionpos=b,                    % sets the caption-position to bottom
	tabsize=4,
	commentstyle=\color{mygreen},    % comment style
	escapeinside={\%*}{*)},          % if you want to add LaTeX within your code
	keywordstyle=\color{blue},       % keyword style
	stringstyle=\color{mymauve}\ttfamily,     % string literal style
	numbers=left, 
	%	frame=single,
	rulesepcolor=\color{red!20!green!20!blue!20},
	% identifierstyle=\color{red},
	language=c
}


\definecolor{mygreen}{rgb}{0,0.6,0}
\definecolor{mymauve}{rgb}{0.58,0,0.82}
\definecolor{mygray}{gray}{.9}
\definecolor{mypink}{rgb}{.99,.91,.95}
\definecolor{mycyan}{cmyk}{.3,0,0,0}

%题目,作者,学校,日期
\title{Paper Reading}
%\subtitle{\fontsize{9pt}{14pt}\textbf{跨临界分岔}}
\author{Speaker: Yichen Lu\quad \newline  \newline \quad }
\institute{\fontsize{8pt}{14pt}}
\date{\today}
\newcommand{\concept}{Paper Reading}

%学校Logo
%\pgfdeclareimage[height=0.5cm]{sustech-logo}{sustech-logo.pdf}
%\logo{\pgfuseimage{sustech-logo}\hspace*{0.3cm}}

\AtBeginSection[]
{
	\begin{frame}<beamer>
	\frametitle{\textbf{Contents}}
	\tableofcontents[currentsection]
\end{frame}
}
% \beamerdefaultoverlayspecification{<+->}
% -----------------------------------------------------------------------------
\begin{document}
% -----------------------------------------------------------------------------
% \frame{\titlepage}

\begin{frame}
    \centering
        \includegraphics[width=\textwidth]{title.png}
\end{frame}

\section{Model}
\begin{frame}
    \begin{multicols}{2}
        
        {
        \footnotesize

        $$
        \begin{aligned}
            \dot{\mathbf{r}}_i&=-\mu \sum_j{\partial _{\mathbf{r}_i}U\left( a_{ij} \right) +\sqrt{2D}\xi_i}\\
            a_{ij}&=\frac{\left| \mathbf{r}_i-\mathbf{r}_j \right|}{\sigma \left( \theta _i \right) +\sigma \left( \theta _j \right)}\\
            \sigma \left( \theta _i \right) &=\sigma _0\frac{1+\lambda \sin \theta _i}{1+\lambda}\\
            U\left( a \right) &=\begin{cases}
                U_0\left( a^{-12}-2a^{-6} \right) ,&		a<1\\
                0,&		\mathrm{otherwise}\\
            \end{cases}\\
            \dot{\theta}_i&=\omega -\sum_j{\left[ \tau \left( a_{ij},\theta _i-\theta _j \right) +\mu _{\theta}\partial _{\theta _i}U\left( a_{ij} \right) \right] +\sqrt{2D_{\theta}}\eta _i}\\
            \tau \left( a,\theta \right) &=\begin{cases}
                \varepsilon \sin \left( \theta \right) ,&		a<1\\
                0,&		\mathrm{otherwise}\\
            \end{cases}
        \end{aligned}
        $$

        \begin{tabular}{ll}
            Symbols& Meanings \\
            \hline
            $\mathrm{r}$ & position \\
            $\theta$ & phase, \\
             & determining the particle size \\
            $U$ & pairwise repulsive potential \\
            $\mu$ & self-propulsion mobility \\
            $D$ & diffusivity \\
            $\xi$ & isotropic Gaussian white noise \\
            $\sigma(\theta)$ & particle size \\
            $\sigma_0$ & largest size \\
            $\lambda<1$ & pulsation amplitude \\
        \end{tabular}
        }

    \end{multicols}

\end{frame}

% -----------------------------------------------------------------------------
\end{document}

