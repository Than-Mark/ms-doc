\documentclass{article}
% \usepackage{xeCJK}
\usepackage{amsmath}
\usepackage{amssymb}
\usepackage{mathrsfs}
\usepackage{xcolor}
\usepackage{bm}
\usepackage{hyperref}
\usepackage{graphicx}
\usepackage{subcaption}
\usepackage{float}
\usepackage{multicol}
\usepackage[ruled,linesnumbered]{algorithm2e}

\bibliographystyle{plain}
\setlength{\parindent}{2em}
\usepackage{geometry}
\geometry{a4paper, left=2.54cm, right=2.54cm, top=3.18cm, bottom=3.18cm}

% 设置文章行距
% \renewcommand{\baselinestretch}{1.5}

% define reference format
\hypersetup{
    colorlinks=true,
    linkcolor=blue,
    urlcolor=blue,
    citecolor=blue,
    linkbordercolor=white
}

\title{\textbf{Chemotactic Chiral Active Matter}}
\author{Yichen Lu}

\begin{document}

\maketitle

\tableofcontents

\newpage
\section{Models}
\subsection{Definitions}
\subsubsection{Self-propelled dynamics}

\begin{subequations}
    \begin{align}
    \dot{x}_i&=v\cos \theta _i\;,
    \\
    \dot{y}_i&=v\sin \theta _i\;,
    \end{align}
    \label{eq:dotxy}
\end{subequations}

\subsubsection{Polar alignment dynamics}
\begin{itemize}
    \item Additive coupling: 
    \begin{equation}
        \label{eq:additionalCouplingDotTheta}
        \dot{\theta}_i=\omega _i+K\sum_{j=1}^{N}f\left( r_{ij} \right)\sin \left( \theta _j-\theta _i+\alpha \right)\;,
    \end{equation}
    \item Mean-field coupling by oscillator number:
    \begin{equation}
        \label{eq:swarmalatorDotTheta}
        \dot{\theta}_i=\omega _i+\frac{K}{N}\sum_{j=1}^N{f}\left( r_{ij} \right) \sin \left( \theta _j-\theta _i+\alpha \right) \;,
    \end{equation}
    which is similar to the swarmalator model.
    % \item Mean-field coupling by oscillator density:
    % \begin{equation}
    %     \dot{\theta}_i=\omega _i+\frac{KL^2}{\pi d_{0}^{2}N}\sum_{j=1}^N{f}\left( r_{ij} \right) \sin \left( \theta _j-\theta _i+\alpha \right) \;,
    % \end{equation}
    % where $d_0$ is the radius of the interaction circle.
\end{itemize}
Here,  $f\left( r_{ij} \right)$ is a function of $r=\left| \mathbf{r}_i-\mathbf{r}_j \right|$, and $K$ is the coupling strength. 
The function $f\left( r \right)$ can be defined as
\begin{enumerate}
    \item $f_H\left( r \right)=H\left( d_0-r \right),\;r_0>0$;
    \item $f_E\left( r \right)=e^{-\frac{r}{d_0}},\;r_0>0$.
\end{enumerate}
The natural frequencies $\omega_i$ are distributed with following two cases:
\begin{enumerate}
    \item \textbf{Single-chiral swarmalators:} The natural frequencies $\omega_i$ are distributed in $U\left( \omega _{\min},\omega _{\max} \right)$ for all swarmalators and $\omega _{\min}\omega _{\max}>0$.
    
    \item \textbf{Double-chiral swarmalators:} The frequencies are distributed in two symmetric uniform distributions, representing two types of chirality. Exactly half of the swarmalators have natural frequencies $\omega_i \sim U\left( \omega _{\min},\omega _{\max} \right)$ and the other half have natural frequencies $\omega_i \sim U\left( -\omega _{\max},-\omega _{\min} \right)$.
\end{enumerate}

\subsubsection{Chemotactic dynamics}

Consider two chemical fields $u\left( \mathbf{r},t \right), v\left( \mathbf{r},t \right)$ that are produced by the ensemble of two symmetrically chiral swarmalators. Swarmalators interact with the chemical field and move towards/against the regions with higher concentration, which can be described by the following equation ($i=1, 2, \dots, N$):
\begin{subequations}
    \begin{align}
        \dot{\mathbf{r}}_{i}^{s}&=v\mathbf{p}\left(\theta_{i}^{s}\right)\\
        \dot{\theta}_{i}^{s}&=\omega _{i}^{s}+\alpha ^{s}\mathbf{p}_{i}^{s}\times \nabla u+\beta ^{s}\mathbf{p}_{i}^{s}\times \nabla v
    \end{align}
\end{subequations}
where $\alpha, \beta _{i}^{s}$ denote the ‘chemotactic’ coupling strength and $\mathbf{p}\left(\theta\right)=(\cos \theta,\sin \theta )$ is the unit vector pointing in the direction of the $i$-th swarmalator, $s\in\{p, n\}$ denotes the two chiral species. Here, we used the notation $\mathbf{a}\times\mathbf{b}=a_1 b_2-a_2 b_1$.

These two fields evolve as
\begin{subequations}
    \begin{align}
    \dot{u}&=k_0\sum_{j=1}^N{\delta \left( \mathbf{r}-\mathbf{r}_{j}^{p} \right)}-k_du+D_u\nabla ^2u\;,\\
    \dot{v}&=k_0\sum_{j=1}^N{\delta \left( \mathbf{r}-\mathbf{r}_{j}^{n} \right)}-k_dv+D_v\nabla ^2v\;,
    \end{align}
\end{subequations}
where $S_+$ and $S_-$ are the sets of two chiral swarmalators, $k_0$ is the production rate, $k_d$ is the decay rate, $D_{u,v}$ are the diffusion coefficients.

\subsubsection{Mixed phase dynamics}
\begin{subequations}
    \begin{align}
        \dot{\mathbf{r}}_i&=v\mathbf{p}_i\\
        \dot{\theta}_i&=\omega _i+\beta _{i}^{u}\mathbf{p}_i\times \nabla u+\beta _{i}^{v}\mathbf{p}_i\times \nabla v+\frac{K}{N}\sum_{j=1}^N{f}\left( \left| \mathbf{r}_j-\mathbf{r}_i \right| \right) \sin \left( \theta _j-\theta _i \right) \;,
    \end{align}
\end{subequations}

\subsubsection{Chemotactic Lotka-Volterra dynamics}
\begin{subequations}
    \begin{align}
        &\mathbf{\dot{r}}_{i}^{1,2}=v\mathbf{p}\left( \theta _{i}^{1,2} \right) -\sum_{s\in \left\{ 1,2 \right\}}{\sum_{j=1}^N{\frac{H\left( r_c-\left| \mathbf{r}_{j}^{s}-\mathbf{r}_{i}^{1,2} \right| \right)}{\left| \mathbf{r}_{j}^{s}-\mathbf{r}_{i}^{1,2} \right|^{\beta}}}}\;,\\
        &\dot{\theta}_{i}^{1,2}=\alpha _{1,2}\left| \nabla c_{1,2} \right|\sin \left( \varphi _{c_{1,2}}-\theta _{i}^{1,2} \right) \;,\\
        &\dot{c}_1=D_1\nabla ^2c_1+c_1\left( k_1-k_2c_2 \right) \sum_{j=1}^N{\delta \left( \mathbf{r}-\mathbf{r}_{j}^{1} \right)}\;,\\
        &\dot{c}_2=D_2\nabla ^2c_2+c_2\left( k_3c_1-k_4 \right) \sum_{j=1}^N{\delta \left( \mathbf{r}-\mathbf{r}_{j}^{2} \right)}\;,\\
    \end{align}
\end{subequations}
where $\varphi _{c_{1,2}}=\arctan \left( \frac{\partial _yc_{1,2}}{\partial _xc_{1,2}} \right) $


\section{Short format coarse graining}
We begin with Eq.~(\ref{eq:additionalCouplingDotTheta}), replacing the finite coupling distance alignment interaction with a pseudopotential (the '$\delta$'-interaction). This substitution is justified when the interaction is sufficiently short-ranged, making the specific shape of the associated interaction potential irrelevant to the dynamics of many swarmalators. The pseudopotential is defined as:
\begin{subequations}
    \begin{align}
        &\mathbf{\dot{r}}_{i}^{c}=v\mathbf{p}\left( \theta _{i}^{c} \right) \;,\\
        &\dot{\theta}_{i}^{c}=\omega _{i}^{c}+K\sum_{j=1}{\delta \left( \mathbf{r}_{j}^{c}-\mathbf{r}_{i}^{c} \right) \sin \left( \theta _{j}^{c}-\theta _{i}^{c} \right)}\notag\\
        &+K\sum_{j=1}{\delta \left( \mathbf{r}_{j}^{b}-\mathbf{r}_{i}^{b} \right) \left[ \sin \left( \theta _{j}^{b}-\theta _{i}^{b}+\alpha _0 \right) -\sin \alpha _0 \right]}\;,
    \end{align}
\end{subequations}
where $c\in\left\{+,-\right\}$ is the chirality of the swarmalator $i$ and $b=+$ if $c=-$ and vice versa.  
Then following \cite{DavidSDean_1996} we derive a continuum equation of motion for the combined $N$-swarmalator probability density
\begin{equation}
    \label{eq:globalContinuityDef}
    \rho ^c\left( \mathbf{r},\theta ,t \right) =\sum_{i=1}{\rho _{i}^{c}\left( \mathbf{r},\theta ,t \right)}\;,
\end{equation}
where $\rho _{i}^{c}\left( \mathbf{r},\theta ,t \right) =\delta \left( \mathbf{r}_{i}^{c}\left( t \right) -\mathbf{r} \right) \delta \left( \theta _{i}^{c}\left( t \right) -\theta \right)$ is the probability density of finding $i$-th swarmalator at position $\mathbf{r}$ with phase $\theta$ and chirality $c$ at time $t$.
Since the deterministic dynamical equation Eq.~ (\ref{eq:totalDynamics}) conserves the number of oscillators with a given natural frequency over time, the distribution function evolves according to a continuity equation of the following form:
\begin{equation}
    \frac{\partial \rho _{i}^{c}}{\partial t}=-\nabla \cdot \left( \rho _{i}^{c}v_{\mathbf{r}} \right) -\frac{\partial }{\partial \theta}\left( \rho _{i}^{c}v_{\theta}^{c,i} \right)\;.
    \label{eq:singleContinuity}
\end{equation}
Here, the velocity fields read
\begin{subequations}
    \begin{align}
        &v_{\mathbf{r}}\left( \mathbf{r},\theta ,t \right) =v\mathbf{p}\left( \theta \right) \;,\\
        &v_{\theta}^{c,i}\left( \mathbf{r},\theta ,t \right) =\omega _{i}^{c}+K\int{\text{d}\phi \rho^{c}\left( \mathbf{r},\phi ,t \right) \sin \left( \phi -\theta \right)}\notag\\
        &+K\int{\text{d}\phi \rho ^{b}\left( \mathbf{r},\phi ,t \right) \left[ \sin \left( \phi -\theta +\alpha _0 \right) -\sin \alpha _0 \right]}\;.
    \end{align}
\end{subequations}
Summing Eq.~(\ref{eq:singleContinuity}) over the $i$ and $c$ indices, and using the definition of the density $\rho ^c$ in Eq.~(\ref{eq:globalContinuityDef}), we obtain  
\begin{equation}
    \label{eq:globalContinuity}
    \begin{aligned}
        &\frac{\partial \rho ^c\left( \mathbf{r},\theta ,t \right)}{\partial t}=-v\mathbf{p}\left( \theta \right) \cdot \nabla \rho ^c\left( \mathbf{r},\theta ,t \right) -\frac{\partial}{\partial \theta}\Omega \left( \mathbf{r},\theta ,t \right)\\
        &+K\frac{\partial}{\partial \theta}\rho ^c\int{\text{d}\phi \rho ^c\left( \mathbf{r},\phi ,t \right) \sin \left( \phi -\theta \right)}\\
        &+K\frac{\partial}{\partial \theta}\rho ^c\int{\text{d}\phi \rho ^b\left( \mathbf{r},\phi ,t \right) \left[ \sin \left( \phi -\theta +\alpha _0 \right) -\sin \alpha _0 \right]}\;,
    \end{aligned}
\end{equation}
where $\Omega \left( \mathbf{r},\theta ,t \right) =\sum_{i=1}{\rho _{i}^{c}\left( \mathbf{r},\theta ,t \right) \omega _{i}^{c}}$.
Spatiotemporal dynamics of the ISS indicates $\forall i,c$, $\rho _{i}^{c}\left( \mathbf{r},\theta ,t \right) \equiv \rho _{\text{ISS}}\left( \mathbf{r},\theta ,t \right)$, which yields
\begin{equation}
    \Omega \left( \mathbf{r},\theta ,t \right) =\rho ^c\left( \mathbf{r},\theta ,t \right) \frac{\left( \omega _{\max}+\omega _{\min} \right)}{2}\;.
\end{equation} 

Transforming Eq.~(\ref{eq:globalContinuity}) to Fourier space, yields an equation of motion for the Fourier modes $\varrho _{k}^{c}\left( \mathbf{r},t \right) =\int{\rho ^c\left( \mathbf{r},\theta ,t \right) \text{e}^{\text{i}k\theta}\text{d}\theta}$ of $\rho^c$:
\begin{equation}
    \begin{aligned}
        \frac{\partial \varrho _{k}^{c}}{\partial t}&=-\frac{v}{2}\left[ \frac{\partial}{\partial x}\left( \varrho _{k+1}^{c}+\varrho _{k-1}^{c} \right) -\mathrm{i}\frac{\partial}{\partial y}\left( \varrho _{k+1}^{c}-\varrho _{k-1}^{c} \right) \right]\\
        &-\left[ \frac{\mathrm{i}k\left( \omega _{\max}+\omega _{\min} \right)}{2}\varrho _{k}^{c}-k^2 \right] \varrho _{k}^{c}\\
        &+\frac{\mathrm{i}Kk}{2\pi}\sum_{m=-\infty}^{\infty}{\varrho _{k-m}^{c}F_{-m}\varrho _{m}^{c}}\\
    \end{aligned}
\end{equation}

\section{Coarse graining}
We now follow the strategy in \cite{DavidSDean_1996} to consider the evolution of the density function for a single particle
\begin{equation}
    \rho _i\left( \mathbf{r},\theta ,\omega ,t \right) =\delta \left( \mathbf{r}_i\left( t \right) -\mathbf{r} \right) \delta \left( \theta _i\left( t \right) -\theta \right) g \left( \omega \right)\;, 
\end{equation}
which denotes the probability of finding a particle at position $\mathbf{r}$, with orientation $\theta$ and natural frequency $\omega$, where $g\left( \omega \right)$ is the time independent swarmalator frequency distribution. The density function $\rho _i$ satisfies the continuity equation, and we shall then demonstrate how one may write a closed equation for the global density
\begin{subequations}
    \begin{align}
        \rho \left( \mathbf{r},\theta ,\omega ,t \right) &=\sum_{i=1}^N{\rho _i\left( \mathbf{r},\theta ,\omega ,t \right)}\;,\\
        \varrho \left( \mathbf{r},\theta ,t \right) &=\int_{-\infty}^{+\infty}{\rho \left( \mathbf{r},\theta ,\omega ,t \right) \mathrm{d}\omega}\;.
        \label{eq:coarseDensitySub2}
    \end{align}
\end{subequations}
The derivation follows a well known argument. Consider an arbitrary function $f$ defined on the coordinate space of the system. Using the definition of the density it is a tautology
that
\begin{equation}
    \label{eq:arbitraryFunction}
    f\left( \mathbf{r}_i\left( t \right) ,\theta _i\left( t \right) ,\omega _i \right) =\iiint{\text{d}\mathbf{r}\text{d}\theta \text{d}\omega \rho _i\left( \mathbf{r},\theta ,\omega ,t \right) f\left( \mathbf{r},\theta ,\omega \right)}\;.
\end{equation}
Expanding the differential equation over the next time step $\delta t$ one obtains
\begin{equation}
    \begin{aligned}
        \frac{\text{d}f\left( \mathbf{r}_i,\theta _i,\omega _i \right)}{\text{d}t}&=\frac{\partial f\left( \mathbf{r}_i,\theta _i,\omega _i \right)}{\partial \mathbf{r}_i}\cdot \frac{\text{d}\mathbf{r}_i}{\text{d}t}+\frac{\partial f\left( \mathbf{r}_i,\theta _i,\omega _i \right)}{\partial \theta _i}\frac{\text{d}\theta _i}{\text{d}t}+\frac{\partial f\left( \mathbf{r}_i,\theta _i,\omega _i \right)}{\partial \omega _i}\frac{\partial \omega _i}{\partial t}\\
        &=\iiint{\text{d}\mathbf{r}\text{d}\theta \text{d}\omega \rho _i\left( \mathbf{r},\theta ,\omega ,t \right) \left( \frac{\partial f\left( \mathbf{r},\theta ,\omega \right)}{\partial \mathbf{r}}\cdot \frac{\text{d}\mathbf{r}}{\text{d}t}+\frac{\partial f\left( \mathbf{r},\theta ,\omega \right)}{\partial \theta}\frac{\text{d}\theta}{\text{d}t}+\frac{\partial f\left( \mathbf{r},\theta ,\omega \right)}{\partial \omega}\frac{\text{d}\omega}{\text{d}t} \right)}\\
        &=\iiint{\text{d}\mathbf{r}\text{d}\theta \text{d}\omega \left( \rho _i\left( \mathbf{r},\theta ,\omega ,t \right) \mathbf{\dot{r}}\cdot \nabla f\left( \mathbf{r},\theta ,\omega \right) +\rho _i\left( \mathbf{r},\theta ,\omega ,t \right) \dot{\theta}\frac{\partial}{\partial \theta}f\left( \mathbf{r},\theta ,\omega \right) \right)}\;.\\
    \end{aligned}
\end{equation}
Re-arranging the above integral by integration by parts we obtain
\begin{equation}
    \label{eq:arbitraryFunctionDt1}
    \frac{\text{d}f\left( \mathbf{r}_i,\theta _i,\omega _i \right)}{\text{d}t}=\iiint{\text{d}\mathbf{r}\text{d}\theta \text{d}\omega f\left( \mathbf{r},\theta ,\omega \right) \left( -\nabla \cdot \left( \rho _i\left( \mathbf{r},\theta ,\omega ,t \right) \mathbf{\dot{r}} \right) -\frac{\partial}{\partial \theta}\left( \rho _i\left( \mathbf{r},\theta ,\omega ,t \right) \dot{\theta} \right) \right)}\;.
\end{equation}
However, from \eqref{eq:arbitraryFunction} we may also deduce
\begin{equation}
    \label{eq:arbitraryFunctionDt2}
    \frac{\text{d}f\left( \mathbf{r}_i,\theta _i,\omega _i \right)}{\text{d}t}=\iiint{\text{d}\mathbf{r}\text{d}\theta \text{d}\omega f\left( \mathbf{r},\theta ,\omega \right) \frac{\partial \rho _i\left( \mathbf{r},\theta ,\omega ,t \right)}{\partial t}}\;.
\end{equation}
Comparing equations \eqref{eq:arbitraryFunctionDt1} and \eqref{eq:arbitraryFunctionDt2} we find (using the fact that $f$ is an arbitrary function) that
\begin{equation}
    \label{eq:continuityEquation}
    \frac{\partial \rho _i\left( \mathbf{r},\theta ,\omega ,t \right)}{\partial t}=-\nabla \cdot \left( \rho _i\left( \mathbf{r},\theta ,\omega ,t \right) \mathbf{\dot{r}} \right) -\frac{\partial}{\partial \theta}\left( \rho _i\left( \mathbf{r},\theta ,\omega ,t \right) \dot{\theta} \right)\;.
\end{equation}
We emphasize that this argument is standard and the only subtlety is that we have not carried out any thermal averaging at this point. Summing equation \eqref{eq:continuityEquation} over the $i$ and using the definition of the density $\rho$ we obtain
\begin{equation}
    \frac{\partial \rho \left( \mathbf{r},\theta ,\omega ,t \right)}{\partial t}=-\nabla \cdot \left( \rho \left( \mathbf{r},\theta ,\omega ,t \right) \mathbf{\dot{r}} \right) -\frac{\partial}{\partial \theta}\left( \rho \left( \mathbf{r},\theta ,\omega ,t \right) \dot{\theta} \right)\;.
\end{equation}
\begin{enumerate}
    \item[(1)] For the case of the phase coupling dynamics, the equation for the density $\rho$ is
    \begin{equation}
        \begin{aligned}
            \frac{\partial \rho \left( \mathbf{r},\theta ,\omega ,t \right)}{\partial t}&=-\nabla \cdot \left( \rho \left( \mathbf{r},\theta ,\omega ,t \right) v\mathbf{p}\left( \theta \right) \right) -\frac{\partial}{\partial \theta}\left( \rho \left( \mathbf{r},\theta ,\omega ,t \right) \left( \omega +G\sum_{j=1}^N{\sin \left( \theta _j-\theta \right) \delta \left( \mathbf{r}_j-\mathbf{r} \right)} \right) \right)\\
            &=-v\mathbf{p}\left( \theta \right) \cdot \nabla \rho \left( \mathbf{r},\theta ,\omega ,t \right) -\omega \frac{\partial}{\partial \theta}\rho \left( \mathbf{r},\theta ,\omega ,t \right) \\
            &-G\frac{\partial}{\partial \theta}\rho \left( \mathbf{r},\theta ,\omega ,t \right) \iiint{\text{d}\mathbf{r}'\text{d}\theta '\text{d}\omega '\rho \left( \mathbf{r}',\theta ',\omega ',t \right) \sin \left( \theta '-\theta \right) \delta \left( \mathbf{r}'-\mathbf{r} \right)}\;,
        \end{aligned}
    \end{equation}
    where $\mathbf{p}\left( \theta \right) =\left( \cos \theta ,\sin \theta \right)$. Then for the density $\varrho$ we have
    \begin{equation}
        \label{eq:coarseDensityAlign}
        \begin{aligned}
            \frac{\partial \varrho \left( \mathbf{r},\theta ,t \right)}{\partial t}&=-v\mathbf{p}\left( \theta \right) \cdot \nabla \varrho \left( \mathbf{r},\theta ,t \right) -\frac{\partial}{\partial \theta}\int_{-\infty}^{+\infty}{\omega \rho \left( \mathbf{r},\theta ,\omega ,t \right) \mathrm{d}\omega}\\
            &-G\frac{\partial}{\partial \theta}\varrho \left( \mathbf{r},\theta ,t \right) \iint{\mathrm{d}\mathbf{r}^{\prime}\mathrm{d}\theta^{\prime}\varrho \left( \mathbf{r}^{\prime},\theta^{\prime},t \right) \sin \left( \theta^{\prime}-\theta \right) \delta \left( \mathbf{r}^{\prime}-\mathbf{r} \right)}\\
        \end{aligned}
    \end{equation}
    \item[(2)] For the case of the chemotactic dynamics, the equation for the density $\rho^s$ is
    \begin{equation}
        \begin{aligned}
            \frac{\partial \rho ^s\left( \mathbf{r},\theta ,\omega ,t \right)}{\partial t}&=-\nabla \cdot \left( \rho ^s\left( \mathbf{r},\theta ,\omega ,t \right) v\mathbf{p}\left( \theta \right) \right) -\frac{\partial}{\partial \theta}\left( \rho ^s\left( \mathbf{r},\theta ,\omega ,t \right) \left( \omega +\alpha ^s\mathbf{p}\left( \theta \right) \times \nabla u+\beta ^s\mathbf{p}\left( \theta \right) \times \nabla v \right) \right)\\
            &=-v\mathbf{p}\left( \theta \right) \cdot \nabla \rho ^s\left( \mathbf{r},\theta ,\omega ,t \right) -\omega \frac{\partial}{\partial \theta}\rho ^s\left( \mathbf{r},\theta ,\omega ,t \right)\\
            &-\frac{\partial}{\partial \theta}\rho ^s\left( \mathbf{r},\theta ,\omega ,t \right) \alpha ^s\left[ \left| \nabla u \right|\sin \left( \theta +\varphi _u \right) +\left| \nabla v \right|\sin \left( \theta +\varphi _v \right) \right]
        \end{aligned}
    \end{equation}
    where $\varphi _c=\mathrm{arg}\left( -\partial _yc+\mathrm{i}\partial _xc \right) , c=u, v$. Then for the density $\varrho ^s$ we have
    \begin{equation}
        \label{eq:coarseDensityChemotactic}
        \begin{aligned}
            \frac{\partial \varrho ^s\left( \mathbf{r},\theta ,t \right)}{\partial t}&=-v\mathbf{p}\left( \theta \right) \cdot \nabla \varrho ^s\left( \mathbf{r},\theta ,t \right) -\frac{\partial}{\partial \theta}\int_{-\infty}^{+\infty}{\omega \rho ^s\left( \mathbf{r},\theta ,\omega ,t \right) \mathrm{d}\omega}\\
            &-\frac{\partial}{\partial \theta}\varrho ^s\left( \mathbf{r},\theta ,t \right) \alpha ^s\left[ \left| \nabla u \right|\sin \left( \theta +\varphi _u \right) +\left| \nabla v \right|\sin \left( \theta +\varphi _v \right) \right]\\
        \end{aligned}
    \end{equation}
\end{enumerate}
Next, let's determine the value of item
\begin{equation}
    \int_{-\infty}^{+\infty}{\omega \rho ^s\left( \mathbf{r},\theta ,\omega ,t \right) \text{d}\omega}\;.
\end{equation}
The uniform distribution of disorder state indicates $g\left( \omega \right) =\left[ 2\left( \omega _{\max}-\omega _{\min} \right) \right] ^{-1}$, which is an $\omega$-independent constant. Then we have
\begin{equation}
    \label{eq:omegaMulInt}
    \int_{-\infty}^{+\infty}{\omega \rho ^s\left( \mathbf{r},\theta ,\omega ,t \right) \text{d}\omega}=\begin{cases}
        \frac{1}{2}\rho ^s\left( \mathbf{r},\theta ,\omega ,t \right) \left( \omega _{\max}^{2}-\omega _{\min}^{2} \right) ,&		\text{Single} \text{Chirality}\\
        0,&		\text{Double} \text{Chirality}\\
    \end{cases}\;.
\end{equation}
Similarly, equation \eqref{eq:coarseDensitySub2} can be rewritten as
\begin{equation}
    \label{eq:coarseDensitySub2Rewrite}
    \varrho ^s\left( \mathbf{r},\theta ,t \right) =\rho ^s\left( \mathbf{r},\theta ,\omega ,t \right) \int_{-\infty}^{+\infty}{\text{d}\omega}=\begin{cases}
        2\left( \omega _{\max}-\omega _{\min} \right) \rho ^s\left( \mathbf{r},\theta ,\omega ,t \right) ,&		\text{Single} \text{Chirality}\\
        0,&		\text{Double} \text{Chirality}\\
    \end{cases}\;.
\end{equation}
Substituting equations \eqref{eq:coarseDensitySub2Rewrite} into \eqref{eq:omegaMulInt}, we obtain 

\subsubsection{Angular Fourier expansion of the phase-space distribution}
% The derivation of the evolution equation for the velocity field is actually much more complicated, and one has to resort to approximation schemes. 
As $\varrho(\mathbf{r},\theta,t)$ is a periodic function of $\theta$, it can be expanded in a Fourier series, defined as
\begin{equation}
    \hat{\varrho}_k(\mathbf{r},t)=\int_{-\pi}^\pi \varrho(\mathbf{r},\theta,t) \mathrm{e}^{\mathrm{i}k\theta}\mathrm{d}\theta\;.
\end{equation}
The inverse Fourier transform is
\begin{equation}
    \label{eq:inverseFourier}
    \varrho (\mathbf{r},\theta ,t)=\frac{1}{2\pi}\sum_{k=-\infty}^{\infty}{\hat{\varrho}_k(\mathbf{r},t)\mathrm{e}^{\mathrm{i}k\theta}\;.}
\end{equation}
In this framework, the uniform distribution $\varrho _0(\mathbf{r},\theta ,t)=\left( 2\pi \right) ^{-1}\varrho _{0}^{*}$ corresponds to $\hat{\varrho}_k(\mathbf{r},\omega,t)=\left( 2\pi \right) ^{-1}\varrho _{0}^{*}$ for $k=0$.

Let us use as a basis of the plane the two orthogonal vectors $\mathbf{p}_1=(1,0)$ and $\mathbf{p}_2=(0,1)$. In order to obtain an evolution equation for the velocity field, we multiply equations \eqref{eq:coarseDensityAlign} and \eqref{eq:coarseDensityChemotactic} by $\mathrm{e}\left(\theta\right)$ and integrate over $\theta$ from $-\pi$ to $\pi$. For equation \eqref{eq:coarseDensityChemotactic}, we obtain ($j=1,2$)
\begin{equation}
    \label{eq:angularFourierInt}
    \frac{\partial}{\partial t}\int_{-\pi}^{\pi}{\mathbf{e}_j\left( \theta \right) \varrho \left( \mathbf{r},\theta ,t \right) \mathrm{d}\theta}+v\sum_{l=1}^2{\frac{\partial}{\partial \mathbf{r}_l}\int_{-\pi}^{\pi}{\mathbf{e}_j\left( \theta \right) \mathbf{e}_l\left( \theta \right) \varrho \left( \mathbf{r},\theta ,t \right) \mathrm{d}\theta}}=\int_{-\pi}^{\pi}{\mathbf{e}_j\left( \theta \right) \left( I_{\mathrm{freq}}+I_{\mathrm{chem}} \right) \mathrm{d}\theta}\;,
\end{equation}
where 
\begin{subequations}
    \label{eq:angularFourierIntSub}
    \begin{align}
        &I_{\mathrm{freq}}=-\frac{\partial}{\partial \theta}\int_{-\infty}^{+\infty}{\omega \rho ^s\left( \mathbf{r},\theta ,\omega ,t \right) \mathrm{d}\omega}\;,\\
        &I_{\mathrm{chem}}=-\frac{\partial}{\partial \theta}\varrho ^s\left( \mathbf{r},\theta ,t \right) \alpha ^s\left[ \left| \nabla u \right|\sin \left( \theta +\varphi _u \right) +\left| \nabla v \right|\sin \left( \theta +\varphi _v \right) \right]\;.
    \end{align}
\end{subequations}
To proceed further, it is convenient to identify complex numbers with two-dimensional vectors, in such a way that $\mathbf{e}(\theta)$ is mapped onto $\mathrm{e}^{\mathrm{i}\theta}$. Then, in the same way, $v\hat{\varrho}_1\left( \mathbf{r},t \right) $ is associated with the momentum field $\mathbf{w}\left( \mathbf{r},t \right) =\rho \left( \mathbf{r},t \right) \mathbf{u}\left( \mathbf{r},t \right) $. Hence, we wish to rewrite equation \eqref{eq:angularFourierInt} in such complex notations. For later use, we shall write it in a slightly more general form, replacing $\mathrm{e}^{\mathrm{i}\theta}$ by $\mathrm{e}^{\mathrm{i}k \theta}$:
\begin{equation}
    \label{eq:angularFourierIntComplex}
    \frac{\partial}{\partial t}\int_{-\pi}^{\pi}{\mathrm{e}^{\mathrm{i}k\theta}\varrho \left( \mathbf{r},\theta ,t \right) \mathrm{d}\theta}+v\sum_{l=1}^2{\frac{\partial}{\partial \mathbf{r}_l}\int_{-\pi}^{\pi}{\mathrm{e}^{\mathrm{i}k\theta}\mathbf{e}_l\left( \theta \right) \varrho \left( \mathbf{r},\theta ,t \right) \mathrm{d}\theta}}=\int_{-\pi}^{\pi}{\mathrm{e}^{\mathrm{i}k\theta}\left( I_{\mathrm{freq}}+I_{\mathrm{chem}} \right) \mathrm{d}\theta}\;.
\end{equation}
Equation \eqref{eq:angularFourierInt} is recovered for $k = 1$, up to the mapping between complex numbers and two-dimensional vectors. The first term on the left-hand side is simply $\partial \hat{\varrho}_k/\partial t$. The second term on the left-hand side can be evaluated as follows: For $l=1,2$ and $k$ integer, let us define the complex quantity $Q_{l}^{\left( k \right)}\left( \mathbf{r},t \right)$ as 
\begin{equation}
    Q_l^{(k)}(\mathbf{r},t)=\int_{-\pi}^\pi\mathrm{d}\theta \mathrm{e}^{\mathrm{i}k\theta}\mathrm{e}_l(\theta)f(\mathbf{r},\theta,t).
\end{equation}
The following relations are then easily obtained:
\begin{equation}
    \begin{aligned}
        Q_{1}^{(k)}(\mathbf{r},t)&=\frac{1}{2}[\hat{f}_{k+1}(\mathbf{r},t)+\hat{f}_{k-1}(\mathbf{r},t)],\\
        Q_{2}^{(k)}(\mathbf{r},t)&=\frac{1}{2\mathrm{i}}[\hat{f}_{k+1}(\mathbf{r},t)-\hat{f}_{k-1}(\mathbf{r},t)].
    \end{aligned}
\end{equation}


The right-hand side of equation \eqref{eq:angularFourierIntComplex} is computed by inserting the Fourier series expansion \eqref{eq:inverseFourier} into equations \eqref{eq:angularFourierIntSub}. 
After a rather straightforward calculation, one finds

% \begin{equation}
%     Q_1^{(k)}(\mathbf{r},t)=\frac{1}{2}[\hat{f}_{k+1}(\mathbf{r},t)+\hat{f}_{k-1}(\mathbf{r},t)],\\Q_2^{(k)}(\mathbf{r},t)=\frac{1}{2\mathrm{i}}[\hat{f}_{k+1}(\mathbf{r},t)-\hat{f}_{k-1}(\mathbf{r},t)].
% \end{equation}


\bibliography{ref}

\end{document}