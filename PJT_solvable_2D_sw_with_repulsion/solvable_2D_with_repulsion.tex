\documentclass{article}
% \usepackage{xeCJK}
\usepackage{amsmath}
\usepackage{amssymb}
\usepackage{mathrsfs}
\usepackage{xcolor}
\usepackage{bm}
\usepackage{hyperref}
\usepackage{graphicx}
\usepackage{subcaption}
\usepackage{float}
\usepackage{csquotes}
\usepackage{multicol}
\usepackage[ruled,linesnumbered]{algorithm2e}

\bibliographystyle{plain}
\setlength{\parindent}{2em}
\usepackage{geometry}
\geometry{a4paper, left=2.54cm, right=2.54cm, top=3.18cm, bottom=3.18cm}

% 设置文章行距
% \renewcommand{\baselinestretch}{1.5}

% define reference format
\hypersetup{
    colorlinks=true,
    linkcolor=blue,
    urlcolor=blue,
    citecolor=blue,
    linkbordercolor=white
}

\title{\textbf{Solvable Two-dimensional Swarmalator Model for Realistic Spatial Interactions}}
\author{Yichen Lu}

\begin{document}

\maketitle

\tableofcontents

\newpage
\section{\label{sec:model}The Model}

O'Keeffe et al. proposed a solvable 2D swarmalator model with periodic boundary conditions:
\begin{subequations}
    \label{eq:totalDynamics}
    \begin{align}
        \dot{x}_i&=\frac{J}{N}\sum_{j=1}^N{\sin \left( x_j-x_i \right) \cos \left( \theta _j-\theta _i \right) \;,}\\
        \dot{y}_i&=\frac{J}{N}\sum_{j=1}^N{\sin \left( y_j-y_i \right) \cos \left( \theta _j-\theta _i \right) \;,}\\
        \dot{\theta}_i&=\frac{K}{N}\sum_{j=1}^N{\sin \left( \theta _j-\theta _i \right) \left( \cos \left( x_j-x_i \right) +\cos \left( y_j-y_i \right) \right) \;,}
    \end{align}
\end{subequations}
which loses the spatial repulsion term. 

Swarmalators have a spatial position $\mathbf{r}_i=\left( x_i, y_i \right)$ and an internal phase $\theta_i$ which evolve according to equations:
\begin{subequations}
    \label{eq:totalDynamics}
    \begin{align}
        \dot{x}_i&=\frac{1}{N}\sum_{j=1}^N{\left[ \sin \left( x_j-x_i \right) \left( 1+J\cos \left( \theta _j-\theta _i \right) \right) -P\sin 2\left( x_j-x_i \right) \right]}\;,\\
        \dot{y}_i&=\frac{1}{N}\sum_{j=1}^N{\left[ \sin \left( y_j-y_i \right) \left( 1+J\cos \left( \theta _j-\theta _i \right) \right) -P\sin 2\left( y_j-y_i \right) \right]}\;,\\
        \dot{\theta}_i&=\frac{K}{N}\sum_{j=1}^N{\sin \left( \theta _j-\theta _i \right) \left( \cos \left( x_j-x_i \right) +\cos \left( y_j-y_i \right) \right)}\;,
    \end{align}
\end{subequations}



\end{document}