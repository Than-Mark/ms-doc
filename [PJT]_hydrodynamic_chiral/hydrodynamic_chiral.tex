\documentclass{article}
\usepackage{xeCJK}
\usepackage{amsmath}
\usepackage{amssymb}
\usepackage{mathrsfs}
\usepackage{xcolor}
\usepackage{bm}
\usepackage{hyperref}
\usepackage{graphicx}
\usepackage{subcaption}
\usepackage{float}
\usepackage{multicol}
\usepackage[ruled,linesnumbered]{algorithm2e}

\setlength{\parindent}{2em}
\usepackage{geometry}
\geometry{a4paper, left=2.54cm, right=2.54cm, top=3.18cm, bottom=3.18cm}

% \twocolumn

% 设置文章行距
\renewcommand{\baselinestretch}{1.5}

% 定义引用格式
\hypersetup{
    colorlinks=true,
    linkcolor=blue,
    urlcolor=blue,
    citecolor=blue,
    linkbordercolor=white
}

\title{\textbf{Two Coupled Swarmalators with Chirality}}

\begin{document}

\maketitle

\section{Reference}
\subsection{Farrell F D C, Marchetti M C, Marenduzzo D, et al. Pattern formation in self-propelled particles with density-dependent motility[J]. Physical review letters, 2012, 108(24): 248101.}
Microscopic dynamics:
\begin{equation}
    \label{eq:microscopicEq}
    \begin{aligned}
        \dot{\mathbf{r}}_i&=v\mathbf{e}_{\theta _i}\;,\\
        \dot{\theta}_i&=\gamma \sum_{j=1}^N{F\left( \theta _j-\theta _i,\mathbf{r}_j-\mathbf{r}_i \right)}+\sqrt{2\epsilon}\tilde{\eta}_i\left( t \right)\;.\\
    \end{aligned}
\end{equation}
The microscopic density of particles at position $\mathbf{r}$ with angle $\theta$ is given by
\begin{equation}
    \label{eq:localDensity}
    f \left( \mathbf{r},\theta\right)=\sum_{i=1}^N{\delta \left( \mathbf{r}-\mathbf{r}_i \right)\delta \left( \theta-\theta _i \right)}\;.
\end{equation}
Using It\^{o}'s formula, a stochastic dynamical equation for the density Eq.~\eqref{eq:localDensity} can be derived:
\begin{equation}
    \label{eq:localDensityDynamics}
    \begin{array}{l}
        \partial _tf\left( \mathbf{r},\theta \right) +\mathbf{e}_{\theta}\cdot \nabla \left[ vf \right]\\
        =\epsilon \frac{\partial ^2f}{\partial \theta ^2}-\frac{\partial}{\partial \theta}\sqrt{2\epsilon f}\eta -\gamma \frac{\partial}{\partial \theta}\int{\mathrm{d}\theta ^{\prime}\mathrm{d}\mathbf{r}^{\prime} f\left( \mathbf{r}^{\prime},\theta ^{\prime} \right)}\times f\left( \mathbf{r},\theta \right) F\left( \theta ^{\prime}-\theta ,\mathbf{r}-\mathbf{r}^{\prime} \right)\;.\\
    \end{array}
\end{equation}
Drop the noise term, and Fourier transform Eq.~\eqref{eq:localDensityDynamics} to get equations of motion for
\begin{equation}
    \label{eq:FourierDensity}
    f_k\equiv \int{f\left( \mathbf{r},\theta \right) e^{ik\theta}\mathrm{d}\theta}\;.
\end{equation}

\section{Our Work}
We replace the finite range alignment interaction by a pseudopotential ($\delta$-interaction) in the model:
\begin{equation}
    \label{eq:ourModel}
    \begin{aligned}
        \dot{\mathbf{r}}_i&=v\mathbf{p}_i\\
        \dot{\theta}_i&=\omega _i+\lambda \sum_{j\ne i}{\delta}\left( \mathbf{r}_j-\mathbf{r}_i \right) \sin \left( \theta _j-\theta _i \right)\\
    \end{aligned}
\end{equation}
where $\mathbf{p}_i=(\cos\theta_i, \sin\theta_i)$. The combined probability density of finding a particle at position $\mathbf{r}_i$ with angle $\theta_i$ is given by
\begin{equation}
    f_i\left( \mathbf{r},\theta \right) =\delta \left( \mathbf{r}-\mathbf{r}_i \right) \delta \left( \theta -\theta _i \right)\;.
\end{equation}
The basic idea is to replace $f_i\omega_i$ by its mean plus a typical fluctuation
\begin{equation}
    \label{eq:ourNoise}
    \omega _if_i\longrightarrow \bar{\omega}f+\sqrt{\Delta _{\omega}f}\eta\;,
\end{equation}
where $\eta$ describes Gaussian random numbers with zero mean and unit variance 
\begin{equation}
    \left\langle\eta(\mathbf{r}, \theta, t) \eta\left(\mathbf{r}^{\prime} \theta^{\prime}, t\right)\right\rangle=\delta\left(\mathbf{r}-\mathbf{r}^{\prime}\right) \delta\left(\theta-\theta^{\prime}\right)\;,
\end{equation}
and $\bar{\omega}, \Delta\omega$ are defined by
\begin{equation}
    \left \langle \omega _i\omega _j \right \rangle =\bar{\omega}^2+\Delta _{\omega}\delta _{ij}
\end{equation}
Physically, we should understand Eq.\eqref{eq:ourNoise} as a local replacement and interpret the above averages as \textbf{mesoscopic} ones
over all particles within the interaction range around a given point $\textrm{x}$ rather than a global average over the whole rotor
ensemble. Accordingly, we allow $\bar{\omega}$ and $\eta$ to fluctuate in space and time. We now define a field $\omega(\mathbf{x}, t)$ representing the deviation from the global (time-independent) average rotation frequency $\bar{\omega}_0=\left< \omega _i \right>$.
\begin{equation}
    \bar{\omega}=\bar{\omega}_0+\omega \left( \mathbf{x},t \right)
\end{equation}

We now expand $\sqrt{f}$ for modest deviations from isotropy as follows
\begin{equation}
    \sqrt{f}=\sqrt{\sum_{k=-\infty}^{\infty}{f_k}\mathrm{e}^{-\mathrm{i}k\theta}}\approx \frac{\sqrt{f_0}}{2}+\frac{1}{2\sqrt{f_0}}\sum_{k=-\infty}^{\infty}{f_k}\mathrm{e}^{-\mathrm{i}k\theta}\;.
\end{equation}

After a long calculation, we find the following equations
\begin{equation}
    \begin{aligned}
        \dot{\rho}&=-v\nabla \cdot \mathbf{w}\\
        \dot{\mathbf{w}}&=(\lambda \rho -2)\frac{\mathbf{w}}{2}-\frac{v}{2}\nabla \rho +\frac{v^2}{2b}\nabla ^2\mathbf{w}-\frac{\lambda ^2}{b}|\mathbf{w}|^2\mathbf{w}\\
        &+\frac{\lambda v}{4b}\left[ 5\nabla \mathbf{w}^2-10\mathbf{w}(\nabla \cdot \mathbf{w})-6(\mathbf{w}\cdot \nabla )\mathbf{w} \right]\\
        &+\omega \mathbf{w}_{\bot}+\frac{v^2\omega}{4b}\nabla ^2\mathbf{w}_{\bot}-\frac{\lambda ^2\omega}{2b}|\mathbf{w}|^2\mathbf{w}_{\bot}\\
        &-\frac{\lambda v\omega}{8b}\left[ 3\nabla _{\bot}\mathbf{w}^2-6\mathbf{w}\left( \nabla _{\bot}\cdot \mathbf{w} \right) -10\left( \mathbf{w}\cdot \nabla _{\bot} \right) \mathbf{w} \right]\\
    \end{aligned}
\end{equation}
where 
\begin{equation}
    \begin{aligned}
        \omega &=\left< \omega _i \right> +\omega \left( \mathbf{x},t \right) +\sqrt{\frac{\Delta _{\omega}}{f}}\eta\\
        b&=2\left( 4+\omega ^2 \right)\\
        \mathbf{w}_{\bot}&=\left( -w_y,w_x \right)\\
        \nabla _{\bot}&=\left( -\partial _y,\partial _x \right)\\
    \end{aligned}
\end{equation}

\end{document}