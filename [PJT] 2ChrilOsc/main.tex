\documentclass{article}
\usepackage{xeCJK}
\usepackage{amsmath}
\usepackage{amssymb}
\usepackage{mathrsfs}
\usepackage{bm}
\usepackage{hyperref}
\usepackage{graphicx}
\usepackage{subcaption}
\usepackage{float}
\usepackage{multicol}
\usepackage[ruled,linesnumbered]{algorithm2e}

\setlength{\parindent}{2em}
\usepackage{geometry}
\geometry{a4paper, left=2.54cm, right=2.54cm, top=3.18cm, bottom=3.18cm}

% 设置文章行距
\renewcommand{\baselinestretch}{1.5}

% 定义引用格式
% \newcommand{\myeqref}[1]{\eqref{#1}式}
% \newcommand{\figref}[1]{图\ref{#1}}
% \newcommand{\tabref}[1]{表\ref{#1}}
\title{\textbf{Two Coupled Oscillators with Chirality}}

\begin{document}

\maketitle

\section{The Model}

\subsection{Raw model}

\begin{eqnarray}
    &&\dot{x}_{1,2}=v\cos \theta _{1,2}\;,\\
    &&\dot{y}_{1,2}=v\sin \theta _{1,2}\;,\\
    &&\dot{\theta}_{1,2}=\omega _{1,2}+\lambda f\left( r \right) \sin \left( \theta _{2,1}-\theta _{1,2} \right)\;,
\end{eqnarray}
where $f\left( r \right)$ is a function of $r=\left| \mathbf{r}_1-\mathbf{r}_2 \right|$, and $\lambda$ is the coupling strength. The function $f\left( r \right)$ can be defined as

\begin{enumerate}
    \item $f\left( r \right) _H=H\left( r_0-r \right),\;r_0>0$;
    \item $f\left( r \right) _P=\left( 1+\frac{r}{r_0} \right) ^{-\frac{1}{r_0}},\;r_0>0$;
    \item $\dots$
\end{enumerate}

\subsection{Model under polar coordinates}

Let
\begin{eqnarray}
    &&x_i=r_i\cos \varphi _i\;,\\
	&&y_i=r_i\sin \varphi _i\;,
\end{eqnarray}
then we have
\begin{eqnarray}
    &&\dot{r}_i=\frac{1}{r_i}\left( x_i\dot{x}_i+y_i\dot{y}_i \right) =v\cos \varphi _i\cos \theta _i+v\sin \varphi _i\sin \theta _i=v\cos \left( \varphi _i-\theta _i \right) \;,\\
	&&\dot{\varphi}_i=\frac{1}{r_{i}^{2}}\left( x\dot{y}-y\dot{x} \right) =\frac{v}{r_i}\left( \sin \varphi _i\cos \theta _i-\cos \varphi _i\sin \theta _i \right) =\frac{v}{r_i}\sin \left( \varphi _i-\theta _i \right) \;.
\end{eqnarray}
Introduce $\alpha_i=\varphi_i-\theta_i$, $\Delta \theta =\theta _2-\theta _1$, $\Delta \varphi =\varphi _1-\varphi _2$, $\Delta \omega =\omega _2-\omega _1$, then the model becomes

\begin{eqnarray}
    &&\dot{r}_{1,2}=v\cos \alpha _{1,2}\;,\\
    &&\dot{\alpha}_{1,2}=\frac{v}{r_{1,2}}\sin \alpha _{1,2}-\omega _{1,2}\mp \lambda f\left( r \right) \sin \Delta \theta\;,\\
    &&\Delta \dot{\varphi}=\frac{v}{r_1}\sin \alpha _1-\frac{v}{r_2}\sin \alpha _2\;,\\
    &&\Delta \dot{\theta}=\Delta \omega -2\lambda f\left( r \right) \sin \Delta \theta\;,
\end{eqnarray}
where
\begin{equation}
    \begin{aligned}
        r&=\sqrt{\left( r_1\cos \varphi _1-r_2\cos \varphi _2 \right) ^2+\left( r_1\sin \varphi _1-r_2\sin \varphi _2 \right) ^2}\\
        &=\sqrt{r_{1}^{2}+r_{2}^{2}-2r_1r_2\cos \Delta \varphi}\\
    \end{aligned}\;.
\end{equation}
So the function $f\left( r \right)$ can be defined as $f\left( r_1,r_2,\Delta \varphi \right)$.

\subsection{Single direction driving}

Assuming that $\dot{\theta}_2=\omega _2, \alpha_2=\cfrac{\pi\text{sgn}\omega_2}{2}$, which means that the second oscillator is rotating around the origin with a constant angular velocity $\omega_2$, and the first oscillator is driven by the second one. Then the model becomes
\begin{eqnarray}
    &&\dot{r}_1=v\cos \left( \Delta \varphi +\Delta \theta +\alpha _2 \right)\;,\\
    &&\Delta \dot{\varphi}=\omega _2-\frac{v}{r_1}\sin \left( \Delta \varphi +\Delta \theta +\alpha _2 \right)\;,\\
    &&\Delta \dot{\theta}=\Delta \omega -\lambda f\left( r_1,\Delta \varphi \right) \sin \Delta \theta\;.
\end{eqnarray}
When $2\lambda f\left( r \right) \geqslant \left| \Delta \omega \right|$, the system has fixed points $\mathbf{x}$, which are
\begin{eqnarray}
    &&r_1=\frac{v}{\omega _2}\;,\\
	&&\Delta \varphi =C_{\Delta\varphi}\;,\\
	&&\Delta \theta =C_{\Delta\theta}\;,
\end{eqnarray}
where $C_{\varphi}$ and $C_{\theta}$ are constants determined by the initial conditions. Linearizing the governing equations yields
\begin{equation}
    M=\left[ \begin{matrix}
        0&		-v\sin \alpha _1&		0&		0\\
        -\frac{v}{r_{1}^{2}}\sin \alpha _1&		\frac{v}{r_1}\cos \alpha _1&		0&		0\\
        \frac{v}{r_{1}^{2}}\sin \alpha _1&		-\frac{v}{r_1}\cos \alpha _1&		0&		0\\
        -\lambda f_{r_1}\sin \Delta \theta&		0&		-\lambda f_{\Delta \varphi}\sin \Delta \theta&		-\lambda f\cos \Delta \theta\\
    \end{matrix} \right] 
\end{equation}
where $f_{r_1}=\cfrac{\partial f}{\partial r_1}$ and $f_{\Delta \varphi}=\cfrac{\partial f}{\partial \Delta \varphi}$. Evaluating $M$ at the fixed points results in
\begin{equation}
    M=\left[ \begin{matrix}
        0&		-v\text{sgn} \omega _2&		0&		0\\
        -\frac{\omega _{2}^{2}}{v}\text{sgn} \omega _2&		0&		0&		0\\
        \frac{\omega _{2}^{2}}{v}\text{sgn} \omega _2&		0&		0&		0\\
        -\lambda f_{r_1}\left( \mathbf{x} \right) \sin C_{\theta}&		0&		-\lambda f_{\Delta \varphi}\left( \mathbf{x} \right) \sin C_{\theta}&		-\lambda f\left( \mathbf{x} \right) \cos C_{\theta}\\
    \end{matrix} \right] 
\end{equation}

\subsection{For $f\left( r \right) = f\left( r \right) _P$}

When $f\left( r \right) = f\left( r \right) _P$, the partial derivatives of $f\left( r \right)$ are
\begin{eqnarray}
    &&\frac{\partial f}{\partial r_1}=r_1g\left( r_1,\Delta \varphi \right)\;,
    \\
    &&\frac{\partial f}{\partial \Delta \varphi}=\frac{v^2}{\omega _{2}^{2}}g\left( r_1,\Delta \varphi \right) \sin \Delta \varphi\;,
\end{eqnarray}
where
\begin{eqnarray}
    &&g\left( r_1,\Delta \varphi \right) =-\frac{f^{1+r_0}\left( r_1,\Delta \varphi \right)}{r_{0}^{2}\sqrt{r_{1}^{2}-\frac{2v^2\cos \Delta \varphi}{\omega _{2}^{2}}+\frac{v^2}{\omega _{2}^{2}}}}\;,
    \\
    &&f\left( r_1,\Delta \varphi \right) =\left( 1+\frac{\sqrt{r_{1}^{2}-2v^2\cos \Delta \varphi /\omega _{2}^{2}+v^2/\omega _{2}^{2}}}{r_0} \right) ^{-\frac{1}{r_0}}\;.
\end{eqnarray}
At the fixed points, the matrix $M$ becomes
\begin{equation}
    M=\left[ \begin{matrix}
        0&		-v\text{sgn} \omega _2&		0&		0\\
        -\frac{\omega _{2}^{2}}{v}\text{sgn} \omega _2&		0&		0&		0\\
        \frac{\omega _{2}^{2}}{v}\text{sgn} \omega _2&		0&		0&		0\\
        -\lambda \frac{v}{\omega _2}g\left( \mathbf{x} \right) \sin C_{\theta}&		0&		-\lambda \frac{v^2}{\omega _{2}^{2}}g\left( \mathbf{x} \right) \sin C_{\Delta \varphi}\sin C_{\theta}&		-\lambda f\left( \mathbf{x} \right) \cos C_{\theta}\\
    \end{matrix} \right] 
\end{equation}
where
\begin{eqnarray}
    &&g\left( \mathbf{x} \right) =-\frac{\left| \omega _2 \right|f^{1+r_0}\left( \frac{v}{\omega _2},C_{\Delta \varphi} \right)}{vr_{0}^{2}\sqrt{2-2\cos C_{\Delta \varphi}}}\;,\\
    &&f\left( \mathbf{x} \right) =\left( 1+\frac{v\sqrt{2-2\cos C_{\Delta \varphi}}}{\left| \omega _2 \right|r_0} \right) ^{-\frac{1}{r_0}}\;.
\end{eqnarray}
The eigenvalues of $M$ are
\begin{eqnarray}
    &&\lambda _{1,2}=\pm \sqrt{\frac{g\left( \mathbf{x} \right)\lambda v^2\sin C_{\Delta \theta}}{\left| \omega _2 \right|}-\frac{\omega _{2}^{4}}{\left| \omega _2 \right|^2}}\;,
    \\
    &&\lambda _3=-f\left( \mathbf{x} \right)\lambda \cos C_{\Delta \theta}\;,
    \\
    &&\lambda _4=0\;.
\end{eqnarray}

\end{document}