\documentclass[12pt, oneside]{ctexbook}
\usepackage{xeCJK}
\usepackage[margin=3cm]{geometry}
\usepackage{indentfirst}
\setlength{\parindent}{2em}
\setCJKmainfont{FandolHei}

\begin{document}

\section{Swarmalator dynamics under global interaction \newline 在全局作用下的Swarmalator动力学}

    
{\noindent 
$\alpha$被设置为1,相当于对空间距离做了单位化。而$\beta$设置为2,这满足上面提到的条件,在相同的距离下,空间斥力应该比吸引力更大,否则会发生碰撞。
}
另外在相位上,相似空间距离的相互作用函数G被设置为了空间绝对距离的倒数,也就是距离越小,相互作用越大。

相似相位的空间吸引力又由Fatt来决定,并且不考虑相似相位对空间上斥力的影响Frep,也就是说Frep恒等于1。在Fatt=1+Jcos(theta)中,如果J为正,那么相近相位的个体就会相互吸引,反之如果是负,相反相位的才会相互吸引,

把J的取值范围设置为(-1, 1)这样$\mathrm{F}_att$的取值是严格正的。相位之间的相互作用遵循Kuramoto模型,也就是相位差的正弦函数,相位差越大,相互作用力越小,相位差越小,相互作用力越大。

在这里swarmalators设置相同的速度与振动频率,且全部为0。得到的模型如图(7)和(8)

在这里,swarmalators之间在空间和相位上的相互作用是全连接的,也就是说每个个体都会受到其他个体的影响。

可变参数J和K的取值决定了模型在长期动力学上的状态,分别是图1所示的5个,分别是“静态同步”,“静态异步”,“静态相位波”,“分裂相位波”,“活跃相位波”

在稳定状态下,swarmalators的空间运动和相位都静止。在K>0时,都是“静态同步”状态。相位的耦合强度K的值取值为正,使得个体的相位差越来越小,最终同步。

在K<0时,有“静态异步”,“分裂相位波”,“活跃相位波”三种状态,在后两种状态下,swarmalators表现出在空间中持续运动,并且他们的相位也在保持变化,所以他们被称为活跃状态。

在没有相位耦合的情况下,也就是K=0时,swarmalators的相位被锁定在初值上,对于J>0,swarmalators会在2维平面内跑列成环形。

swarmalators的初始位置从$[-1, 1]\times[-1, 1]$的范围内随机选取,初始相位从$[0, 2\pi]$的范围内随机选取。

\subsection{序参量}

序参量是一个复数,它的模长代表了群体的同步程度,而它的辐角代表了群体的相位。
这里$\Phi$被是swarmalators的平均相位,$R\in \left[ 0, 1 \right]$衡量了单元之间的相位相干性。

在稳定同步状态中,所有的swarmalators的相位都是相同的,R的值为1。在另外四种状态中,R严格小于1,因为相位没有同步。

相位$\theta_i$和空间角度$\phi_i$的相关性由另一个序参量衡量

根据定义,如果S的值非零,那么相位和空间角度是相关的。在静态相位波状态下,相位与空间角度完全相关,我的理解是因为相位和空间角度都是周期性的,此时S的值为1。当K值从0减小到负值时,相关性减小,最终在静态异步状态下达到最小值0,swarmalators的相位均匀分布在0到2π之间。

静态异步在超过临界相耦合强度$K_c=-1.2J$后失去了稳定性。S值非零,但严格小于1

在分裂相波状态下,振荡器分裂成簇,在每个簇内,波动器移动,相位在其平均值附近振荡。 Swarmalators不会从一个集群移动到另一个集群,也不会形成不相交的集群。这类集群的数量和每个集群内的数量取决于初始条件。

在活跃相位波状态下,团簇的形成消失,并在相位和空间角度上执行有规律的圆周运动。

\subsection{相位相似性对空间吸引和排斥力的影响}

在静态相位波状态下,相位与空间角度完全相关。因此,这种状态被称为环相波。

\subsection{在外力作用下的swarmalators}

和同步中队外力感染研究类似,swarmalators在外力干扰的情况下也会表现出一些性质,当刺激器振幅增加时,靠近中心的swarmalators开始与外部频率$\Omega$同步。随着F的增加,经历了从部分同步到完全同步的相变(5个状态)。在静态同步状态下,刺激位置附近的振荡器与刺激位置同步,在刺激位置周围形成小簇。

\section{Dynamics under local interaction \newline 在局部作用下的动力学}

在前一节中,我们研究了它们之间的空间和相位相互作用是全局的波动器。但是,在现实中,大多数现实世界的多代理系统都表现出代理之间的局部邻居交互。

现在swarmalators只在一个有限的范围r内和它们的邻居相互作用。当r趋于无穷时,局部相互作用变成全局相互作用,表现出和之前相似的5种状态。当考虑r的有限值时,系统显示出有趣的结果。当$r>D_\infty$时,观察到许多稳定状态的复制。这种出现多个副本的原因是:

一旦一组swarmalators聚集在半径为r的圆形区域,在那里它们与其他物体相距最小距离r,它们就独立地进化为一个独立的群体.遵循和r趋向于于无穷,即全局相互作用的情况相同的动力学

因此,两组之间的最小分离是$D_\infty$。对于$r<D_\infty$,发现了静态同步和静态异步状态的异常版本。在r的一定时间间隔内(1 < r < 1.8),在静态相位波状态下出现了条状模式。如果我们考虑swarmalators之间的局部吸引相位耦合,可以观察到出现多个不相同的静态簇。

\section{Competitive phase interaction 竞争性质的相位相互作用}

到目前为止,我们已经讨论了振荡器的长期状态的结果,其中单元之间的相位耦合要么是正的(吸引的),要么是负的(排斥的)。但许多系统的各单元间的耦合特性往往更为复杂。

在这篇文献种,第i个和第j个swarmalators之间的相位耦合强度Kij从双峰分布中随机选择。其中Ka和Kr分别为吸引耦合强度和排斥耦合强度,p为吸引耦合的概率。

\end{document} 