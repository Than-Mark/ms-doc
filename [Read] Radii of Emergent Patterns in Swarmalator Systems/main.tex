\documentclass[10pt,aspectratio=43,mathserif,table]{beamer} 
%  设置为 Beamer 文档类型,设置字体为 10pt,长宽比为16:9,数学字体为 serif 风格
\batchmode
\usepackage{graphicx}
\usepackage{subfigure}
\usepackage{fontspec}
\usepackage{multicol}

% \setmainfont{Harding Text Web Regular Regular.ttf}
\usepackage{diagbox} % 表头斜线分区
\usepackage{unicode-math}
\usefonttheme{serif}

\usetheme{Berlin} %主题
\setbeamertemplate{page number in head/foot}[pagenumber]
%\usecolortheme{sustech} %主题颜色

\usepackage[ruled,linesnumbered]{algorithm2e}

\usepackage{fancybox}
\usepackage{xcolor}
\usepackage{listings}

\usepackage{booktabs}
\usepackage{colortbl}

\newcommand{\Console}{Console}
\lstset{ %
	backgroundcolor=\color{white},   % choose the background color
	basicstyle=\footnotesize\rmfamily,     % size of fonts used for the code
	columns=fullflexible,
	breaklines=true,                 % automatic line breaking only at whitespace
	captionpos=b,                    % sets the caption-position to bottom
	tabsize=4,
	commentstyle=\color{mygreen},    % comment style
	escapeinside={\%*}{*)},          % if you want to add LaTeX within your code
	keywordstyle=\color{blue},       % keyword style
	stringstyle=\color{mymauve}\ttfamily,     % string literal style
	numbers=left, 
	%	frame=single,
	rulesepcolor=\color{red!20!green!20!blue!20},
	% identifierstyle=\color{red},
	language=c
}


\definecolor{mygreen}{rgb}{0,0.6,0}
\definecolor{mymauve}{rgb}{0.58,0,0.82}
\definecolor{mygray}{gray}{.9}
\definecolor{mypink}{rgb}{.99,.91,.95}
\definecolor{mycyan}{cmyk}{.3,0,0,0}

%题目,作者,学校,日期
\title{Paper Reading: Radii of Emergent Patterns in Swarmalator Systems}
%\subtitle{\fontsize{9pt}{14pt}\textbf{跨临界分岔}}
\author{Speaker: Yichen Lu\quad \newline  \newline \quad }
\institute{School of Mathematical Science}
\date{\today}
\newcommand{\concept}{Paper Reading}

%学校Logo
%\pgfdeclareimage[height=0.5cm]{sustech-logo}{sustech-logo.pdf}
%\logo{\pgfuseimage{sustech-logo}\hspace*{0.3cm}}

\AtBeginSection[]
{
	\begin{frame}<beamer>
	\frametitle{\textbf{Contents}}
	\tableofcontents[currentsection]
\end{frame}
}
% \beamerdefaultoverlayspecification{<+->}
% -----------------------------------------------------------------------------
\begin{document}
% -----------------------------------------------------------------------------
% \frame{\titlepage}

% \frame{\titlepage}
\begin{frame}
    \begin{figure}
        \centering
        \includegraphics[width=\textwidth]{front.jpg}
    \end{figure}
\end{frame}


\begin{frame}
    Model equations:
    $$
    \begin{aligned}
        \dot{\boldsymbol{x}}_i&=\frac{1}{N}\sum_{j\ne i}^N{e_{ij}d_{ij}^{\alpha}\left( 1+J\cos \theta _{ij} \right) -e_{ij}d_{ij}^{\beta}}\\
        \dot{\theta}_i&=\frac{K}{N}\sum_{j\ne i}^N{d_{ij}^{\gamma}\sin \theta _{ij}}\\
    \end{aligned}
    $$
    where 
    $d_{ij}=\left\| \boldsymbol{x}_j-\boldsymbol{x}_i \right\|$, 
    $\theta _{ij}=\theta _i-\theta _j$, 
    $e_{ij}=\cfrac{1}{d_{ij}}(\boldsymbol{x}_j-\boldsymbol{x}_i)$.

    Choosing $\alpha = 0$ and $\beta=\gamma=-1$ gives the original model.
\end{frame}

\begin{frame}
    In the converged state of the static patterns, the entities no longer move, so we have $\dot{\boldsymbol{x}}_i = 0$, which yields:

    $$
    r_{\mathrm{out}}^{\alpha}\underset{\mathrm{Attraction} A}{\underbrace{\sum_{j\ne i}^N{e_{ij}g_{ij}^{\alpha}\left( 1+J\cos \theta _{ij} \right)}}}=r_{\mathrm{out}}^{\beta}\underset{\mathrm{Repulsion} R}{\underbrace{\sum_{j\ne i}^N{e_{ij}g_{ij}^{\alpha}}}}
    $$

    where $r_{\mathrm{out}}=\underset{1\le i\le N}{\max}\left\| \boldsymbol{x}_i-\bar{\boldsymbol{x}} \right\| $, 
    $g_{ij}=\frac{d_{ij}}{r_{\mathrm{out}}}$.

\end{frame}

\begin{frame}
    \begin{columns}
        \begin{column}{0.4\textwidth}
            \begin{figure}
                \centering
                \includegraphics[width=\textwidth]{f2a.png}
            \end{figure}
        \end{column}
        \begin{column}{0.6\textwidth}
            \footnotesize
            \begin{itemize}
                \item Pattern center $\bar{\boldsymbol{x}}$ is positioned at $(1, 0)$ and a swarmalator is present at the origin.
                \item using polar coordinates $(r, \phi)$, hecnce $e_{ij}=\left( \cos \phi ,\sin \phi  \right)$ and $g_{ij}=r$.
                \item $
                l_{O1}\left( \phi \right) =2\cos \phi =\frac{2\sin \phi \cos \phi}{\sin \phi}=\frac{\sin 2\phi}{\sin \phi}
                $
            \end{itemize}
            
        \end{column}
    \end{columns}
    
    $ $

    Due to symmetry, $\sum\nolimits_{j\ne i}^N{e_{ij,y}g_{ij}^{\alpha}}=0$, $\sum\nolimits_{j\ne i}^N{e_{ij,y}g_{ij}^{\beta}}=0$,  
    which yields 
    \vspace{-0.2cm}
    $$r_{\mathrm{out}}^{\alpha}\sum\nolimits_{j\ne i}^N{e_{ij,x}g_{ij}^{\alpha}\left( 1+J\cos \theta _{ij} \right)}=r_{\mathrm{out}}^{\beta}\sum\nolimits_{j\ne i}^N{e_{ij,x}g_{ij}^{\alpha}}$$

    Next, the authors consider the limiting case $N \rightarrow \infty$
    and rewrite the sums as integrals ($\mathrm{d}x\mathrm{d}y = r\mathrm{d}r\mathrm{d}\phi$)
    \vspace{-0.3cm}
    $$
    r_{\mathrm{out}}^{\alpha -\beta}=\frac{\mathrm{Repulsion}}{\mathrm{Attraction}}=\frac{\int_{-\frac{\pi}{2}}^{\frac{\pi}{2}}{\cos \phi \int_0^{l_{O1}\left( \phi \right)}{r^{1+\beta}\mathrm{d}r\mathrm{d}\phi}}}{\int_{-\frac{\pi}{2}}^{\frac{\pi}{2}}{\cos \phi \int_0^{l_{O1}\left( \phi \right)}{\left( 1+J\cos \theta _{ij} \right) r^{1+\alpha}\mathrm{d}r\mathrm{d}\phi}}}
    $$
\end{frame}

\begin{frame}
    The solution in the first integration: 
    $$
    \int_0^{l_{O1}\left( \phi \right)}{r^{1+m}dr}=\frac{1}{2+m}l_{O1}^{2+m}\left( \phi \right) =\frac{\left( 2\cos \phi \right) ^{2+m}}{2+m}
    $$
    Supposed $\theta_{ij}$ is a constant, we have
    $$
    \begin{aligned}
        r_{\mathrm{out}}^{\alpha -\beta}&=\frac{\frac{2^{2+\beta}}{2+\beta}\int_{-\frac{\pi}{2}}^{\frac{\pi}{2}}{\cos ^{3+\beta}\phi \mathrm{d}\phi}}{\frac{\left( 1+J\cos \theta _{ij} \right) 2^{2+\alpha}}{2+\alpha}\int_{-\frac{\pi}{2}}^{\frac{\pi}{2}}{\cos ^{3+\alpha}\phi \mathrm{d}\phi}}\\
        &=\frac{2^{\beta -\alpha}\left( 2+\alpha \right) \int_{-\frac{\pi}{2}}^{\frac{\pi}{2}}{\cos ^{3+\beta}\phi \mathrm{d}\phi}}{\left( 1+J\cos \theta _{ij} \right) \left( 2+\beta \right) \int_{-\frac{\pi}{2}}^{\frac{\pi}{2}}{\cos ^{3+\alpha}\phi \mathrm{d}\phi}}\\
    \end{aligned}
    $$



\end{frame}

\begin{frame}
    \begin{itemize}
        \item \textbf{Static Sync.} In Static Sync State, $J\cos \theta _{ij}=J\,\,\forall \left( i, j \right)$, For $\alpha = 0$ and $\beta=-1$, we have
        $$
            r_{\mathrm{out}}=\frac{\int_{-\frac{\pi}{2}}^{\frac{\pi}{2}}{\cos \phi \frac{\sin \left( 2\phi \right)}{\sin \phi}\mathrm{d}\phi}}{\left( 1+J \right) \frac{1}{2}\int_{-\frac{\pi}{2}}^{\frac{\pi}{2}}{\cos \phi \frac{\sin ^2\left( 2\phi \right)}{\sin ^2\phi}\mathrm{d}\phi}}=\frac{3\pi}{8\left( 1+J \right)}
        $$


        \item \textbf{Static Async.} In Static Async State, $J\cos \theta _{ij}=0\,\,\forall \left( i, j \right)$, For $\alpha = 0$ and $\beta=-1$, we have
        $$
            r_{\mathrm{out}}=\frac{\int_{-\frac{\pi}{2}}^{\frac{\pi}{2}}{\cos \phi \frac{\sin \left( 2\phi \right)}{\sin \phi}\mathrm{d}\phi}}{ \frac{1}{2}\int_{-\frac{\pi}{2}}^{\frac{\pi}{2}}{\cos \phi \frac{\sin ^2\left( 2\phi \right)}{\sin ^2\phi}\mathrm{d}\phi}}=\frac{3\pi}{8}\approx 1.178
        $$
    \end{itemize}
\end{frame}




% -----------------------------------------------------------------------------
\end{document}