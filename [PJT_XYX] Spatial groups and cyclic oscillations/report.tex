\documentclass{article}
\usepackage{xeCJK}
\usepackage{amsmath}
\usepackage{amssymb}
\usepackage{mathrsfs}

\setlength{\parindent}{2em}
\usepackage{geometry}
\geometry{a4paper, left=2.54cm, right=2.54cm, top=3.18cm, bottom=3.18cm}

\begin{document}

\section{相位-取向关联的集群振子系统的动力学研究}

\textbf{$\Delta$ 聚焦点}
\begin{enumerate}
    \item 环态及其相变
    \item 环态的解域与相位同步的关系
    \item 数值结果的细致讨论, 分类
    \item 必要的理论分析与估计
\end{enumerate}

分析点
\subsection{单个粒子的运动问题(无相互作用)}

$$
\begin{cases}
	\begin{array}{c}
	\Delta x\left( t \right) =v\cos \theta \Delta t\\
	\Delta y\left( t \right) =v\sin \theta \Delta t\\
\end{array}\rightarrow \begin{array}{c}
	\dot{x}=v\cos \theta\\
	\dot{y}=v\sin \theta\\
\end{array}\\
	\dot{\theta}_i=\omega _i\rightarrow \theta _i\left( t \right) =\omega _it\\
	v=\sqrt{\dot{x}_{i}^{2}+\dot{y}_{i}^{2}}=v\left( constant \right)\\
\end{cases}
$$

运动半径 $=\ ?$

$$
\begin{cases}
	x_i\left( t \right) =x_i\left( 0 \right) -\frac{v}{\omega _i}\sin \omega _it\\
	y_i\left( t \right) =y_i\left( 0 \right) -\frac{v}{\omega _i}\cos \omega _it\\
\end{cases}
$$

$$
\Rightarrow \left( x_i-x_{i}^{0} \right) ^2+\left( y_i-y_{i}^{0} \right) ^2=\left( \frac{v}{\omega _i} \right) ^2
$$

每个粒子的运动轨迹是一个圆,圆心为 $\left( x_i^0,y_i^0 \right)$,半径为 $\frac{v}{\omega _i}$

\subsection{考虑相互作用$\lambda$,耦合距离$d_0$ ($\{A_ij\}$,注意是时变的)}

\begin{itemize}
    \item 看空间聚集过程
    \item 空间尺度也考虑进去
\end{itemize}

粒子数$N$,$L\times L\sim \sqrt{\frac{L\times L}{N}}\sim \frac{L}{\sqrt{N}}$

\begin{enumerate}
    \item 每个单粒子的运动空间尺度, $\cfrac{v}{\omega_i}$
    \item 耦合距离 $d_0$
    \item 粒子平均间距 $\cfrac{L}{\sqrt{N}}$
\end{enumerate}

$$
d_0\sim \frac{L}{\sqrt{N}}\rightarrow d_0\ll \frac{L}{\sqrt{N}}, d_0\gg \frac{L}{\sqrt{N}}
$$

低频粒子:$d_0\ll \frac{v}{\omega _i}$
高频粒子:$d_0\gg \frac{v}{\omega _i}$

在计算空间pattern同时,还要跟踪每个粒子的相速度$\dot{\theta}_i(t)$

\end{document}